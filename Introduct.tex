\section{Introduction}
Automatically extracting events from natural text remains a challenging task in information extraction. Among diverse types of event extraction systems, the extraction task proposed by Automatic Content Extraction (ACE) \cite{doddington2004automatic} is the most popular framework, which defines two main terminologies: \textbf{trigger} and \textbf{argument}. The former is the word that most clearly expresses the occurrence of an event. The latter is a phrase that serves as a participant or attribute with a specific role in an event.

However, constructing training data for ACE task is expensive. First, linguists are required to summarize a large amount of text to elaborately design templates about potential arguments for each event type. Second, rules should be explicitly stated to guide annotators. In spite of detailed guidelines, there is still disagreement among human annotators about what should (not) be regarded as triggers/arguments. 
For example, can a prepositional phrase or a portion of a word trigger an event, e.g., \textit{in prison} triggers an \emph{arrest} event, 
or, \textit{ex} in \textit{ex-husband} triggers a \emph{divorce} event?
Besides, ACE event extraction systems remain two major limitations:   single-token trigger labeling, and one type for one event.

%The aforementioned drawbacks of ACE event extraction systems motivate us to 
It would be interesting to see (1) can we automatically build a dataset for event extraction without experts involved? 
and (2) can we have an event extractor that handles more realistic scenarios, e.g.,  when trigger annotations are unavailable, or events with more than one type.

First, we observe that structured knowledge bases (KB) often store
%we draw the inspiration from our observations on knowledge base (KB). A knowledge base 
complex structured information, which share similar structures with ACE event definitions and a particular entry of which usually implies the occurrence of certain events.
%  and some of which shares a highly similar structure with event templates defined in ACE. Thus, any sentence that contains some participants and attributes in a particular KB entry is likely to imply an event in some way. 
On the other hand, recent studies \cite{mintz2009distant,zeng2015distant} have demonstrated the effectiveness of KB as distant supervision for binary relation extraction.
%, which is another important task in information extraction. 
However, there are two major challenges when leveraging KB to event extraction: first, event structures are more complex than binary relations. They can be represented as $\langle event\_type, argument_1, \ldots, argument_n\rangle$, which are n-ary relations with various numbers of arguments. Second, there is no trigger information in any existing knowledge base. Therefore, to refine the original distant supervision assumption in relation extraction, we investigate different hypotheses for better data quality and quantity. Among these hypotheses, the vital one is that, for a particular event type, there are some \textbf{key arguments}, which together can trigger an event and distinguish the corresponding type with other event types. We utilize Freebase as our knowledge base and Wikipedia articles as text for data generation. According to Mintz et al. \shortcite{mintz2009distant}, because a major source of Freebase is the tabular data from Wikipedia, making it a natural fit with Freebase. Figure~\ref{fig:3} illustrates examples of sentences annotated by our algorithm.

Second, unlike previous studies focus on tasks defined by ACE evaluation framework \cite{ahn2006stages,li2013joint,chen2015event,nguyen2016joint}, we propose a novel event extraction paradigm with key arguments to characterize an event type. We split event extraction into two sequence labeling subtasks, namely event detection and argument detection. Following the fashion in other sequence labeling tasks, like POS tagging and NER \cite{huang2015bidirectional,lample2016neural}, we utilize a LSTM-CRF model to label key arguments and non-key arguments in the sentences separately. However, LSTM-CRF is limited as event structures are not sequential and there are strong dependencies between key arguments. We therefore reformulate the hypotheses as constraints, and apply linear integer programming to output multiple optimal label sequences. We conduct both manual and automatic evaluation for the detected events. 

In this paper, we intend to exploit existing structured knowledge base, e.g., Freebase, as distant supervision to automatically annotate event structures from plain text without human annotator. And we propose a novel event extraction paradigm that relieve the cost of event trigger annotations at same time. Under this paradigm, we present a LSTM-CRF model with post inference to extract general events and multi-type events on the generated corpus, which is demonstrated effective by both manual and automatic evaluation.
