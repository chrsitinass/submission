\section{Related Work}
Event extraction is one of the fundamental research topics in information extraction and natural language understanding. There are diverse descriptions of event extraction tasks defined by different programs (MUC \cite{grishman1996message}, ACE \cite{doddington2004automatic}, ERE \cite{song2015light} and TACKBP \cite{mitamura2015event}), all of which can be summarized as template-filling-based event extraction. Nearly all approaches on these corpora use a supervised paradigm, and highly rely on the human-annotated data. We typically divide them into feature-based methods and neural-network-based methods. 

Feature-based methods usually rely on a variety of elaborately features. They aim to exploit different feature extraction strategies and evaluate feature contributions to the classification. Li et al. \shortcite{li2013joint} jointly learn trigger labeling and argument labeling using structured perceptron to capture both local and global features of triggers and arguments. 

Neural-network-based methods are free of hard feature engineering and error propagation from external NLP tools. Two types of neural works have been employed. Chen et al. \shortcite{chen2015event} propose a convolutional neural network (CNN) with a dynamic multi-pooling layer to capture sentence-level features better. Nguyen et al. \shortcite{nguyen2016joint} propose a bidirectional RNN with various memory matrices to jointly learn triggers and arguments, which benefits from both joint models and neural network models. However, to our knowledge, LSTM-CRF models have not been applied in earlier studies.

In contrast to these prior systems focused on small human-labeled corpus, Huang et al. \shortcite{huang2016liberal} propose a novel Liberal Event Extraction paradigm which automatically discovers event schemas and extract events simultaneously from any unlabeled corpus. 

Freebase is a typical resource for distant supervision in binary relation extraction \cite{mintz2009distant,zeng2015distant}. However, we can not simply apply their data labeling strategies to event extraction, as event structures is much more like a n-ary relation extraction.
