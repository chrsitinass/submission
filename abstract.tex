\begin{abstract}
Event extraction is an important technology that underpins many text mining applications. It is a task of extracting
events of certain types and their corresponding arguments of different roles from unstructured texts such as news
articles and tweets. Existing event extraction systems are typically built through applying supervised learning over
expert-annotated datasets with a limited set of event types. Constructing a high-quality training dataset currently
requires heavy human involvement. As a result, the generated dataset is often small and can only cover a narrow scope
of event types. This limitation of datasets makes the learned event extractors hard to generalize. In this paper, we
propose a novel distant supervision based approach to automatic construct event extraction training data with a wide
range of event types, through leveraging knowledge extracted from structured knowledge bases. \todo{novelty?} To extend
the capability of the current event extraction systems, We develop a novel neural network with ILP-based post inference
committing to support multi-type events and multi-word arguments, making the extraction systems applicable in many more
real-world scenarios. We apply our approach to \todo{xx}. Experimental
results show that our method can \todo{xx}. 


%often supervised and rely on expert-annotated datasets, with limited event types.
%%such as ACE and ERE event extraction frameworks.
%However, designing and constructing these
%high-quality corpora, usually with limited size and coverage of event types,  is costly, which
%makes learned extractors hard to generalize.  With the essence of distant supervision,
%%Inspired by some Freebase schemas which share similar structures with ACE event templates,
%we investigate the possibilities of automatic construction of training data for various event types
%with the help of structured knowledge bases.
%%the following problems in this paper: can we generate a feasible dataset for event extraction with Freebase automatically and is it possible to extract events on this dataset.
%%We first propose four hypotheses based on our observation and produce our dataset accordingly. Then,
%We further propose a novel neural network with ILP-based post inference committing to
%handling two challenges in event extraction: multi-type events and multi-word arguments.
%Both automatic and manual evaluations demonstrate that it is possible to learn to extract various  events, according to existing knowledge bases, without human-annotated training data.
\end{abstract} 